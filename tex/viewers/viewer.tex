\documentclass[11pt]{article}
\usepackage{enumerate,amsmath,amssymb,amsthm}%,fancyhdr}
\usepackage{setspace}
\usepackage{sectsty}
\usepackage[margin = 1 in]{geometry}
\usepackage{graphicx}
\footskip  0.25in
\usepackage{booktabs}
\usepackage{etoolbox}
\usepackage{stackengine}
\usepackage{listings}
\usepackage{subfiles}
\usepackage{subcaption}
\usepackage{caption}
\usepackage{tabularx}
\usepackage{multirow}
\usepackage{xcolor}
\usepackage{verbatim}
\usepackage{dsfont}
\usepackage{array}
\usepackage{siunitx}

\DeclareMathOperator*{\argmax}{arg\,max}

%%%%%%%%%%%%%% bibliography
\usepackage{natbib}
\bibliographystyle{chicago}

\begin{document}

\section*{Viewer document and importing tables into tex}

This is what a typical viewer looks like for me; a simple tex document where I can input tables and figures. Note that calling these tables in the actual paper is exactly the same "shell".

\begin{table}[!htbp]
\centering
\setlength{\tabcolsep}{2pt}
\def\sym#1{\ifmmode^{#1}\else\(^{#1}\)\fi}
\caption{Main regressions}
\footnotesize{

\begin{tabular}[t]{lccc}
\toprule
  & Simple & Expanded & Pro\\
\midrule
Age & $151.495$ & $124.453$ & $-72.298$\\
 & ($198.006$) & ($221.682$) & ($187.837$)\\
Sex &  & $70.410$ & $-209.527$\\
 &  & ($249.324$) & ($214.293$)\\
BMI &  &  & $1033.386$***\\
 &  &  & ($216.534$)\\
\midrule
N & $50$ & $50$ & $50$\\
Adj. $R^2$ & $-0.01$ & $-0.03$ & $0.30$\\
\bottomrule
\multicolumn{4}{l}{\rule{0pt}{1em}* p $<$ 0.1, ** p $<$ 0.05, *** p $<$ 0.01}\\
\end{tabular}


}
\raggedright
\noindent
\footnotesize{Notes: }
\end{table}

\end{document}